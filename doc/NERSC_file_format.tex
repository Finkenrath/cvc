\chapter{The NERSC file format}

The NERSC file format for gauge configurations is as follows.

The first part of the file consists of a header in ASCII format describing the gauge configuration stored in the
second part of the file in binary format.

\section{Header}

The beginning of the header is marked by the keyword \texttt{BEGIN\_HEADER}, the end is marked
by \texttt{END\_HEADER}. All lines between these two have the format \\
\texttt{keyword = value}.\\

The number of lines in the header (between mandatory beginning and ending keyword) can vary. There is a
set of minimum keywords that have to appear; these include \texttt{DIMENSION\_[1234]}, \ldots (?)


A a generic header will thus look as follows. \\
\texttt{
BEGIN\_HEADER \\
HDR\_VERSION = 1.0 \\
DATATYPE = 4D\_SU3\_GAUGE \\
STORAGE\_FORMAT = 1.0 \\
DIMENSION\_1 = 32 \\
DIMENSION\_2 = 32 \\
DIMENSION\_3 = 32 \\
DIMENSION\_4 = 64 \\
PLAQUETTE = 0.59371401 \\
LINK\_TRACE = -0.00007994 \\
CHECKSUM = 5523B964 \\
BOUNDARY\_1 = PERIODIC \\
BOUNDARY\_2 = PERIODIC \\
BOUNDARY\_3 = PERIODIC \\
BOUNDARY\_4 = PERIODIC \\
ENSEMBLE\_ID = Nf0\_b6p00\_L32T64 \\
SEQUENCE\_NUMBER = 10000 \\
FLOATING\_POINT = IEEE32BIG \\
CREATION\_DATE = 06:58:49 11/07/2011  \\
CREATOR = hopper06 \\
END\_HEADER \\
}

\section{Binary data}

We now give the space-time, spinor and color structure in the binary data block.
We consider the case where the for each color matrix only the first two lines are stored, such the third line must
be reconstructed. The binary data part of the file thus contains 
$LX\cdot LY\cdot LZ\cdot T\cdot 4 \cdot 2 \cdot 3 \cdot 2 \cdot \mathrm{prec}$ bytes, where prec is 4 (single precision) or
8 (double precision).


To access an element $U_\mu(w)[a,b]$ of color matrix $U$ at lattice site $w = (x,y,z,t)$ in direction 
$\mu=\hat{x}(0),\,\hat{y}(1),\,\hat{z}(2),\,\hat{t}(3)$ and with $a=0,1$ and $b=0,1,2$ we must use the offset
\begin{equation}
  \mathrm{offset} = ( ( ( t\cdot LZ + z ) \cdot LY + y ) \cdot LX + x ) \cdot 48 + \mu\cdot 12 + 2\cdot (3\cdot a + b)
\end{equation}
for the real part and $\mathrm{offset}+1$ for the imaginary part.

With the label $\hat{i}$, $i=x,y,z,t$, we associate the direction that is dictated by one of the standard choices
of Dirac matrices, $\gamma^{j}$, $j=1,2,3,4$, such that $j=1 \Leftrightarrow i=x$ etc..

Note that the spacetime indices run $(t,z,y,x)$, whereas the directions run $(\hat{x},\hat{y},\hat{z},\hat{t})$.
